\documentclass[12pt]{article}

\usepackage[utf8]{inputenc}
\usepackage{setspace}
\usepackage[letterpaper]{geometry}

\author{Timoth\'ee Poisot \and Philippe Desjardins-Proulx \and Dominique Gravel}
\title{A general model for coevolution in spatialized interaction networks}

\usepackage{kpfonts}
\usepackage{biblatex}
\bibliography{library}

\begin{document}
	
\maketitle\doublespacing

\begin{abstract}
	In this working paper, we outline the basic ingredients for a
	phenomenological model of coevolution in spatial interaction networks.
	This model relies on developments from the insular biogeography theory and
	the metacommunity theory.
\end{abstract}

\section{Introduction}

\section{Ingredients of the model}

The metapopulation framework is ...

Building on these previous results, TTIB built a trophic model of island
biogeography. This model achieves the overlap between ecological network
theory and biogeography, but lacks an evolutionary component. It is the goal
of this paper to descibe how the TTIB model can be coupled to simple
evolutionary rules, to obtain a phenomenological model of evolution in
spatialized networks.

\section{Speciation and intra-specific drift}

In our model, each species is defined by a centroïd $C_S$ in the niche space.
Each population of this species has a given position $C_P$, which allows
calculating how far this population is from its species centroïd. When
$\mathrm{d}(C_S,C_P)$ becomes superior to a given treshold, the isolated
population initiates a new species.

\section{Rules for interaction}

The existence of an interaction follows the same rules as in the niche model
of WILLIAMS. Each species is identified by its position $n$ on a niche space,
which can be composed of as many arbitrary continuous quantitative niche axes
as needed. On each axis, each species has a centroïd $c$, and a range $r$,
meaning that each species will interact with any other species whose trait
value falls within $n+c\pm r$.

\section{Conclusion}

\printbibliography

\end{document}