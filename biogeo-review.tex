\documentclass[11pt,oneside]{scrartcl}

\usepackage{geometry}			
\geometry{letterpaper}
\usepackage{canoniclayout} 
\usepackage[parfill]{parskip}
\usepackage{setspace}

\usepackage{graphicx}
\usepackage{booktabs}
\usepackage[utf8]{inputenc}

\usepackage{pgfplots}
\usepackage{pgfplotstable}
\pgfplotsset{compat=newest}
\usetikzlibrary{shapes,backgrounds}
\usepgfplotslibrary{groupplots}

\usepackage[doi=false,isbn=false,url=false]{biblatex}
\bibliography{library}

\title{The evolutionary biogeography of species interactions networks}
\author{Timoth\'ee Poisot \and Philippe Desjardins-Proulx \and Carlos Meli\`an \and Dominique Gravel}

\begin{document}

\maketitle
\begin{abstract}
	\ldots 
\end{abstract}
\doublespacing


\section{Why}

\begin{itemize}

\item Parallel development of literature on evolution in interactive systems and
food webs in space: time to overlap them, but more than the overlap: Venn
diagram

\item Difficult to tell what the theory will look like
	\begin{itemize}
		\item lack of data
		\item no patterns
	\end{itemize}

\end{itemize}

\begin{figure}[htbp]
   \centering
   \def\firstcircle{(0,0) circle (3cm)}
\def\secondcircle{(2,3.4) circle (3cm)}
\def\thirdcircle{(4,0) circle (3cm)}

\begin{tikzpicture}\sffamily
    \begin{scope}[fill opacity=0.5]
        \fill[red!40] \firstcircle;
        \fill[green!40] \secondcircle;
        \fill[blue!40] \thirdcircle;
        \draw \firstcircle node[below] {};
        \draw \secondcircle node [above] {};
        \draw \thirdcircle node [below] {};
    \end{scope}
    %% Draw the labels
    \draw (5,0) node {\Large Coevolution \cite{Thompson1994a}};
    \draw (2,4.4) node {\Large Biogeography};
    \draw (-1,0) node {\Large Networks \cite{Pascual2006}};
    
    %% What we don't know
    \draw (2,1.2) node {\bfseries\Large ???};
    
    %% References about what we know
    \draw (3.3,2) node {\Large\protect\cite{Thompson2005}};
    
    \draw (-3.5,4.5) node [fill=black!10] (netbiogeo) {\Large \emph{e.g.} \protect\cite{Gravel2011c,Massol2011,Supp2012}};
    \path [->] (netbiogeo) edge [very thick] (0.3,2);
    
    \draw (2,-4.5) node [fill=black!10] (netcoevo) {\Large \emph{e.g.} \protect\cite{Loeuille2005,PoisotPRSLB2012}};
    \path [->] (netcoevo) edge [very thick] (2,-0.7);
    
\end{tikzpicture}
   \caption{example caption}
   \label{fig:venn}
\end{figure}

\section{Processes and mechanisms}

\begin{itemize}
	\item networks
	\item biogeography 
	\item coevolution
	\item figure with arrows between similar mechanisms
	\item Right scale for the mechanisms
		\begin{itemize}
			\item individual
			\item trait
			\item genetic architecture
			\item provides the unification between scales of organization
		\end{itemize}
\end{itemize}

\section{Patterns}

\begin{itemize}
	\item scaling of patterns from ind to pop to comm
	\item not necessarily the same structure at different scales // diff. mecha.
	\item ecosystems
\end{itemize}

\section{What to do?}

\subsection{Data}

\begin{itemize}
	\item Individual data with traits
	\item Genetic architecture
	\item Explicitely sampling different locations
\end{itemize}

\subsection{Models}

\begin{itemize}
	\item Box with the different models
	\item New methods for modelling 1
	\item New methods for modelling 2
\end{itemize}

\section{Conclusions}

\begin{itemize}
	\item \ldots
\end{itemize}

\printbibliography

\begin{enumerate}
	\item Venn diagram
	\item Different levels of data resolution: intricate triangles like the WOL figure
	\item ISI WoL for the overlap between circles
\end{enumerate}

\end{document}