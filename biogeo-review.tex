\documentclass[11pt,oneside]{article}

\usepackage{geometry}			
\geometry{letterpaper}
\usepackage[parfill]{parskip}
\usepackage{setspace}

\usepackage{graphicx}
\usepackage{booktabs}
\usepackage[utf8]{inputenc}

\usepackage{pgfplots}
\usepackage{pgfplotstable}
\pgfplotsset{compat=newest}
\usetikzlibrary{shapes,backgrounds}
\usepgfplotslibrary{groupplots}

\usepackage[style=tree]{biblatex}
\bibliography{library}

\title{The evolutionary biogeography of species interactions networks}
\author{Timoth\'ee Poisot \and Philippe Desjardins-Proulx \and Carlos Meli\`an \and Dominique Gravel}

\begin{document}

\maketitle
\begin{abstract}
	\ldots 
\end{abstract}
\onehalfspacing\clearpage


\section{Why}

In the most recent years, understanding the evolution of species interaction
networks in space emerged as a challenging task \parencite{Pillai2011}, to some
extent because of the lack of an integrative theory \parencite{Urban2008}. Yet,
over the last fifteen years, ecologists and evolutionists developed three bodies
of literature upon which such a theory can be based. First, we developed tools
to analyse the structure of species interactions, in simple and complex
communities. Second, we uncovered the mechanisms involved in species coevolution
\parencite{Thompson1994a}, including how coevolutionary dynamics should be
affected by space and environmental heterogeneity \parencite{Thompson2005}.
Finally, several contrasting hypotheses to explain species distribution
patterns in space were proposed \parencite{O'Dwyer2010}. However, the overlap
between these three fields is still limited (Fig.~\ref{f:venn}), and there is
no overarching theory allowing to merge them into a more predictive framework.

\begin{figure}[htbp]
   \centering
   \def\firstcircle{(0,0) circle (3cm)}
\def\secondcircle{(2,3.4) circle (3cm)}
\def\thirdcircle{(4,0) circle (3cm)}

\begin{tikzpicture}\sffamily
    \begin{scope}[fill opacity=0.5]
        \fill[red!40] \firstcircle;
        \fill[green!40] \secondcircle;
        \fill[blue!40] \thirdcircle;
        \draw \firstcircle node[below] {};
        \draw \secondcircle node [above] {};
        \draw \thirdcircle node [below] {};
    \end{scope}
    %% Draw the labels
    \draw (5,0) node {\Large Coevolution \cite{Thompson1994a}};
    \draw (2,4.4) node {\Large Biogeography};
    \draw (-1,0) node {\Large Networks \cite{Pascual2006}};
    
    %% What we don't know
    \draw (2,1.2) node {\bfseries\Large ???};
    
    %% References about what we know
    \draw (3.3,2) node {\Large\protect\cite{Thompson2005}};
    
    \draw (-3.5,4.5) node [fill=black!10] (netbiogeo) {\Large \emph{e.g.} \protect\cite{Gravel2011c,Massol2011,Supp2012}};
    \path [->] (netbiogeo) edge [very thick] (0.3,2);
    
    \draw (2,-4.5) node [fill=black!10] (netcoevo) {\Large \emph{e.g.} \protect\cite{Loeuille2005,PoisotPRSLB2012}};
    \path [->] (netcoevo) edge [very thick] (2,-0.7);
    
\end{tikzpicture}
   \caption{The integration between theories about species interaction,
   coevolution, and biogeography, is a required step to take in order to reach
   a biogeographic understanding of how species interaction network evolve
   over spatial extents.}
   \label{f:venn}
\end{figure}

Here, we argue that while a theory for the evolutionary biogeography of
species interactions could be built by increasing overlap between these three
fields, there is more to it than that. We identify the need to more closely
investigate mechanisms involved in species interactions and evolution at
different scales of organization (trait, individual, population, community),
and emphasize the need for additional data so as to ground this theoretical
development in empirical results.

Despite the conceptual advance that this paper represents, it is still difficult
to define the form of such an integrative theory. Identifying patterns
against which to match theoretical predictions is difficult, because as we show,
there is a lack of dataset whose design would allow making inferences on the
processes responsible for shaping the evolution of spatial networks. As we
belive that, failing what it may be highly speculative, a theory allowing to
make such predictions should be better grounded in data, we conclude this paper
by proposing new ways to approach the sampling of species interactions, allowing
a better understanding of the underlying evolutionary mechanisms.

% Could we make a list of a few big arguments/points we're trying to make?
% Are we going to argue *for* an individual-based perspective or just put the 
% question out there? (I'm writing something on the community as a group of
% individuals, but it might be worth making the point here *if* you think it's
% appropriate.)
%
% -- PhDP

\section{Processes and mechanisms}

In this section, we review processes and mechanisms belonging to the three
bodies of litterature, likely to act in the evolution of SINs in space, and
discuss which should be part of an integrative theory.

The first three paragraphs describe roughly what we know about each of the three
theories, and how it relates to space / evolution. The last paragraph is
essentially an appeal to make individuals / populations the relevant unit of
observation and modelling. All the mechanisms acting at these scales provide a
unification between both the theories and the scales of organization

\begin{itemize}
	\item Current state of : networks theory
        % Do we cover all networks: food web, mutualistic, etc...?
        % -- PhDP
	\item Current state of : biogeography
        % List of major models:
        %  * TTIB
        %  * O'Dwyer
        %  * ...
        % -- PhDP
	\item Current state of : coevolution
	\item (figure with arrows between similar mechanisms -- tim)
	\item Right scale for the mechanisms
		\begin{itemize}
			\item traits
                        % Shipley et al. It's really worth discussing to some
                        % extend.
                        % -- PhDP.
			\item individual
                        % IMO the right scale. I think it's fairly easy to argue
                        % that many pf the incompatibiltiies between ecology and
                        % evolution are naturally resolved when considering
                        % intra-specific diversity.
                        % -- PhDP
                        \item population
                        % All population-based approaches (MacArthur,
                        % Lotka-Volterra, Wagner, etc...).
                        % -- PhDP
                        \item community
                        % There's an interesting point that some properties of
                        % communities cannot be derived from first principle
                        % (e.g.: Ulanowicz), so the community in itself might be
                        % the right scale.
                        % -- PhDP
			\item genetic architecture
                        % It doesn't seem like a "scale" in the same sense as
                        % the other scales. Nobody ever studied patterns of
                        % diversity as a collection of genetic architectures
                        % (but people do study communities as colelctions of
                        % traits). I would prefer discussing genetic
                        % architecture as something that could be interegrated
                        % in an individual-based approach.
                        % -- PhDP
			\item provides the unification between scales of organization
		\end{itemize}
\end{itemize}

\section{Patterns}

\begin{itemize}
    \item How do patterns of interaction \ evolution scale from individuals to
        communities? Which patterns scale up, and which don't?
    \item not necessarily the same structure at different scales : which mechanisms will scale up ?
    \item do we have predictions for ecosystems (nutrient cycling, \ldots)
\end{itemize}

% I'm very interested in everything related to the individual-community scale,
% so I'd like to contribute to this part.
% -- PhDP

\section{What to do?}

\subsection{Data}

The goal of this section is to outline what kind of data we need, and why

\begin{itemize}
	\item Individual data with traits
	\item Genetic architecture
	\item Explicitely sampling different locations
\end{itemize}

\subsection{Models}

Phil (mostly) will compare different models and what they lack / have 

\begin{itemize}
	\item Box with the different models
	\item New methods for modelling 1
	\item New methods for modelling 2
\end{itemize}

% Do I have a paragraph on the methodological issues (computational complexity
% of individual-based approaches, etc?). I'm just asking, we have tons of things
% to discuss and I understand why it might be better to ignore this complicated
% subject.
% -- PhDP

\section{Conclusions}



\printbibliography

\cleardoublepage

\textbf{List of figures}
\begin{enumerate}
	\item Different levels of data resolution: triangles like the WOL figure
	\item ISI WoL for the overlap between circles
\end{enumerate}

\end{document}
