\documentclass[11pt,oneside]{article}

\usepackage{geometry}			
\geometry{letterpaper}
\usepackage[parfill]{parskip}
\usepackage{setspace}

\usepackage{graphicx}
\usepackage{booktabs}
\usepackage[utf8]{inputenc}

\usepackage{pgfplots}
\usepackage{pgfplotstable}
\pgfplotsset{compat=newest}
\usetikzlibrary{shapes,backgrounds}
\usepgfplotslibrary{groupplots}

\usepackage[doi=false,isbn=false,url=false]{biblatex}
\bibliography{library}

\title{The evolutionary biogeography of species interactions networks}
\author{Timoth\'ee Poisot \and Philippe Desjardins-Proulx \and Carlos Meli\`an \and Dominique Gravel}

\begin{document}

\maketitle
\begin{abstract}
	\ldots 
\end{abstract}
\onehalfspacing\clearpage


\section{Why}

In the most recent years, understanding the evolution of species interaction
networks in space emerged as a challenging task \parencite{Pillai2011}, to some
extent because of the lack of an integrative theory \parencite{Urban2008}. Yet, over the last fitfteen
years, ecologists and evolutionists developped three bodies of litterature upon
which such a theory can be based. First, we developped tools to analyse the
structure of species interactions, in simple and complex communities. Second, we
uncovered the mechanisms involved in species coevolution
\parencite{Thompson1994a}, including how coevolutionary dynamics should be
affected by space and environmental heterogenetiy \parencite{Thompson2005}.
Finally, several contrasted hypothesis to explain species distribution patterns
in space were proposed. As of now, there is still relatively little overlap
between these three fields (Fig.~\ref{f:venn}). Here, we argue that while a
theory for the evolutionary biogeography of species interactions can be built by
increasing overlap between these three fields, there is more to in than that. We
identify the need to more closely investigate mechanisms involved in species interactions and
evolution at different scales of organization (individual, population, community), and \ldots 

Despite the conceptual advance that this paper represents, it is still difficult
to precisely outline what the theory will ressemble. Identifying patterns
against which to match theoretical predictions is difficult, because as we show,
there is a lack of dataset whose design would allow making inferences on the
processes responsible for shaping the evolution of spatial networks. As we
belive that, failing what it may be highly speculative, a theory allowing to
make such predictions should be better grounded in data, we conclude this paper
by proposing new ways to approach the sampling of species interactions, allowing
a better understanding of the underlying evolutionary mechanisms.

\begin{figure}[htbp]
   \centering
   \def\firstcircle{(0,0) circle (3cm)}
\def\secondcircle{(2,3.4) circle (3cm)}
\def\thirdcircle{(4,0) circle (3cm)}

\begin{tikzpicture}\sffamily
    \begin{scope}[fill opacity=0.5]
        \fill[red!40] \firstcircle;
        \fill[green!40] \secondcircle;
        \fill[blue!40] \thirdcircle;
        \draw \firstcircle node[below] {};
        \draw \secondcircle node [above] {};
        \draw \thirdcircle node [below] {};
    \end{scope}
    %% Draw the labels
    \draw (5,0) node {\Large Coevolution \cite{Thompson1994a}};
    \draw (2,4.4) node {\Large Biogeography};
    \draw (-1,0) node {\Large Networks \cite{Pascual2006}};
    
    %% What we don't know
    \draw (2,1.2) node {\bfseries\Large ???};
    
    %% References about what we know
    \draw (3.3,2) node {\Large\protect\cite{Thompson2005}};
    
    \draw (-3.5,4.5) node [fill=black!10] (netbiogeo) {\Large \emph{e.g.} \protect\cite{Gravel2011c,Massol2011,Supp2012}};
    \path [->] (netbiogeo) edge [very thick] (0.3,2);
    
    \draw (2,-4.5) node [fill=black!10] (netcoevo) {\Large \emph{e.g.} \protect\cite{Loeuille2005,PoisotPRSLB2012}};
    \path [->] (netcoevo) edge [very thick] (2,-0.7);
    
\end{tikzpicture}
   \caption{The integration between theories about species interaction, coevolution, and biogeography,
    is a required step to take in order to reach a biogeographic understanding of how species interaction network evolve over
    spatial extents.}
   \label{f:venn}
\end{figure}

\section{Processes and mechanisms}

In this section, we review processes and mechanisms belonging to the three
bodies of litterature, likely to act in the evolution of SINs in space, and
discuss which should be part of an integrative theory.

\begin{itemize}
	\item networks
	\item biogeography 
	\item coevolution
	\item (figure with arrows between similar mechanisms)
	\item Right scale for the mechanisms
		\begin{itemize}
			\item individual
			\item trait
			\item genetic architecture
			\item provides the unification between scales of organization
		\end{itemize}
\end{itemize}

\section{Patterns}

\begin{itemize}
	\item scaling of patterns from ind to pop to comm
	\item not necessarily the same structure at different scales // diff. mecha.
	\item ecosystems
\end{itemize}

\section{What to do?}

\subsection{Data}

\begin{itemize}
	\item Individual data with traits
	\item Genetic architecture
	\item Explicitely sampling different locations
\end{itemize}

\subsection{Models}

\begin{itemize}
	\item Box with the different models
	\item New methods for modelling 1
	\item New methods for modelling 2
\end{itemize}

\section{Conclusions}



\printbibliography

\cleardoublepage

\textbf{List of figures}
\begin{enumerate}
	\item Different levels of data resolution: triangles like the WOL figure
	\item ISI WoL for the overlap between circles
\end{enumerate}

\end{document}