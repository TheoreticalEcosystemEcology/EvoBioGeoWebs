\documentclass[11pt,oneside]{scrartcl}

\usepackage{geometry}			
\geometry{letterpaper}
\usepackage{canoniclayout} 
\usepackage[parfill]{parskip}
\usepackage{setspace}

\usepackage{graphicx}
\usepackage{booktabs}
\usepackage[utf8]{inputenc}

\usepackage{pgfplots}
\usepackage{pgfplotstable}
\pgfplotsset{compat=newest}
\usepgfplotslibrary{groupplots}

\usepackage[doi=false,isbn=false,url=false]{biblatex}
\bibliography{library}

\title{The evolutionary biogeography of species interactions networks}
\author{Timoth\'ee Poisot \and Philippe Desjardins-Proulx \and Carlos Meli\`an \and Dominique Gravel}

\begin{document}

\maketitle
\begin{abstract}
	\ldots 
\end{abstract}
\doublespacing


\section{Why}
\label{why}

\begin{itemize}

\item Parallel development of literature on evolution in interactive systems and food webs in space: time to overlap them, but more than the overlap: Venn diagram


\item Difficult to tell what the theory will look like

\begin{itemize}

\item lack of data
\item no patterns
\end{itemize}


\end{itemize}

\section{Processes and mechanisms}
\label{processesandmechanisms}

\begin{itemize}
	\item networks
	\item biogeography 
	\item coevolution
	\item figure with arrows between similar mechanisms
	\item Right scale for the mechanisms
		\begin{itemize}
			\item individual
			\item trait
			\item genetic architecture
			\item provides the unification between scales of organization
		\end{itemize}
\end{itemize}

\section{Patterns}

\begin{itemize}
	\item scaling of patterns from ind to pop to comm
	\item not necessarily the same structure at different scales // diff. mecha.
	\item ecosystems
\end{itemize}

\section{What to do?}

\subsection{Data}

\begin{itemize}
	\item Individual data with traits
	\item Genetic architecture
	\item Explicitely sampling different locations
\end{itemize}

\subsection{Models}

\begin{itemize}
	\item Box with the different models
	\item New methods for modelling 1
	\item New methods for modelling 2
\end{itemize}

\section{Conclusions}

\begin{itemize}
	\item \ldots
\end{itemize}

\printbibliography

\begin{enumerate}
	\item Venn diagram
	\item Different levels of data resolution: intricate triangles like the WOL figure
	\item ISI WoL for the overlap between circles
\end{enumerate}

\end{document}